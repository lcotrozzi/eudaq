% !TEX TS-program = pdflatexmk
\input{src/preamble}

% Insert EUDET document number here:
\newcommand*\EUDETnum{EUDAQ User Manual}

% PDF info and link colours
\hypersetup{
  pdftitle={EUDAQ2 User Manual},
  pdfauthor={EUDAQ Development Team},
  pdfsubject={EUDAQ},
  pdfkeywords={EUDAQ, AIDA, EUDET, Software, User Manual},
  colorlinks=true,
  linkcolor=black, % \ref \autoref \pageref
  citecolor=blue, % \cite
  urlcolor=blue, % \url \href (external)
  filecolor=blue, % \href (local file)
}

% Load glossary
\loadglsentries{src/glossary.tex}

% Headers and Footers
\titlehead{\EUDETnum}
\lehead{\EUDETnum}
\lohead{\EUDETnum}
\automark{section}
\rehead{\headmark}
\rohead{\headmark}
\chead{}
\cfoot{\pagemark}

% Title info
\subject{
\begin{center}
\includegraphics[width=0.1\textwidth]{src/images/logo_eudet} \includegraphics[width=0.3\textwidth]{src/images/AIDA_logo_medium} 
\includegraphics[width=0.35\textwidth]{src/images/logo_aida2020}
\vspace{1.5cm}\\
\includegraphics[width=0.7\textwidth]{src/images/eudaq_logo}
\end{center}
}
\title{\Large EUDAQ2 User Manual} 
\author{\normalsize EUDAQ2 Development Team \\
        \normalsize Yi Liu}
\input{src/version.tex} % defines the \CMakeLibVersion command which is
                    % set by CMake to the current EUDAQ version number
                    % (with git hash tag)
\date{\normalsize Last update on September 2017 \\  %\today
for EUDAQ version v2.0.1} %\CMakeLibVersion}

\begin{document}

% Title page
\maketitle
\begin{abstract}
\noindent
This document provides an overview of the EUDAQ software framework,
the data acquisition framework originally developed for use by the
EUDET-type beam telescopes~\cite{telescopesWiki}.
It describes how to install and run the DAQ system and use many of the included utility programs,
and how users may integrate their DAQ systems into the EUDAQ framework
by writing their own: Producer, for integrating the data stream into the acquisition; DataCollector, for merging the muiltple data streams and writing to disk; and DataConverter, for converting data for offline analysis.
\end{abstract}
\newpage

% Contents
% Condense slightly to fit on one page
\begin{spacing}{0.92}
\tableofcontents
\end{spacing}

% Main sections
\include{src/License}
\include{src/Introduction}
\include{src/Installing}
% !TEX root = EUDAQUserManual.tex
\section{Run an Example Setup}
This section will describe running the DAQ system, mainly from the point of view of running an example setup as a demonstration without dedicated hardware.
However, this description can be applied to a detector DAQ system in general.

\subsection{Target Setup}
In this example, a user hardware device is simulated and implemented as an example Producer which can be configured to generate fake data. This works similarly to a real Producer, but does not talk to any real hardware. An example DataCollector is also implemented. \\
The runtime setup consists of the following EUDAQ processes.
\begin{description}
\ttitem{RunControl}
an example RunControl instanced by GUI application.
\ttitem{LogCollector}
a default LogCollector instanced by GUI application.
\ttitem{Producer}
Two example Producers instanced by the CLI command. Both of them produces events at a rate of 1\,Hz, they may have time offset of start run.
\ttitem{DataCollector}
an example DataCollectors instanced by the CLI command.
\ttitem{Monitor}
an example monitor which prints out the asambled Event from the DataCollector in the command line terminal.
\end{description}

\subsection{Preparation}
Some preparation is needed to make sure the environment is set up correctly and
the necessary TCP ports are not blocked before the DAQ can run properly.

\subsubsection{Directories}
If no specified path is passed to EUDAQ (by configuration file or command line parameter), EUDAQ will assume the working folder where executable is started up is writable. Data and log files will be stored in the working folder.

\subsubsection{Init/Config-Files}\label{sec:ConfigFiles}
\texttt{$\ast$.ini}-files for initialization and \texttt{$\ast$.conf}-files for configuration
are text files in a specific format, containing name-value pairs separated into different sections.\footnote{\url{https://en.wikipedia.org/wiki/INI\_file}}
Any text from a \texttt{\#} character until the end of the line is treated as a comment, and
ignored.  
Each section in the config file is delimited by a pair of names seperated by a perioid in square brackets (e.g. \verb@[Producer.Example]@).
The name before peroid represents the type of process to which it applies, the one after peroid is the runtime name of the applied process.
. There is an exception, the section for Run Control is always \verb@[RunControl]@ in which there is no runtime name. 
Within each section, any number of parameters may be specified,
in the form \mbox{\texttt{Name = Value}}.  The EUDAQ native supported parmaters have a common prefix \texttt{EUDAQ\_}.
It is then up to the individual processes how these parameters are interpreted.

During the initialization and configuration, each process gets its section and does not know about the other parts of the ini/conf files.
\myinputlisting[conf]{-30}{user/example/misc/Ex0.ini}

\myinputlisting[conf]{-30}{user/example/misc/Ex0.conf}

\subsubsection{Ports and Firewall}
The different processes communicate between themselves using TCP/IP sockets. The ports can be configured when calling the the processors on the command line (see below). By default, TCP port 44000 is listened to by the RunControl, and TCP port 44002 is listened to by the LogCollector to get connections from clients. The DataCollector will pick a random TCP port to listen to and get incoming data from its connected Producers if there is no specific port assigned explicitly by user.

When all EUDAQ processes run on the same computer, common firewalls will not affect the TCP connections among them. However, running EUDAQ distributively on several computers may have issue from firewall blocking. You will have to open up those TCP ports for incoming connections, temporally shut down the firewall. Shutting down the firewall is operating-system dependent.\\

In this example, all processes will run on the same Linux computer, so we can skip the setup of TCP port.

%%%%%%%%%%%%%%%%%%%%%%%%%%%%%%%%%%%%%%%%%%%%%%%%%%%%%%%%%%%%%%%55
\subsection{Startup}
To start EUDAQ, all of the necessary processes have to be started in the correct order.
The first process must be the Run Control,
since all other processes will attempt to connect to it when they start up.
Then it is recommended to start the Log Collector,
since any log messages it receives may be useful
to help with debugging in case everything does not start as expected.
Finally, the Data Collector, Producers and Monitor can be started in any order you want.

\subsubsection{RunControl}
\label{sec:runcontrol}
There are two versions of the RunControl -- a text-based version \texttt{euCliRun} and a graphical version \texttt{euRun} (see \autoref{fig:RunControl}).

The command line pattern to start up a Log Collector is:
\begin{listing}[mybash]
$[euRun]$ -n {code_name} -a tcp://{listening_port}
\end{listing}

\begin{description}
\ttitem{-n \param{code\_name}}
optional, if it is not specified, default RunControl will be instanced. 
\ttitem{-a \param{listening\_addr}}
optional, \texttt{listening\_port} default value is 44000.
\end{description}

For this example setup, we will startup the euRun GUI which internally instances the default Run Control and make it serve at TCP port 44000:\\
\begin{listing}[mybash]
$[euRun]$ -n Ex0RunControl -a tcp://44000
\end{listing}

After executing the above command, a new GUI windows is shown in \autoref{fig:RunControl}.
\begin{figure}[htb]
  \begin{center}
    \includegraphics[width=0.8\textwidth]{src/images/eurun_ui}
    \caption{The Run Control graphical user interface.}
    \label{fig:RunControl}
  \end{center}
\end{figure}

\paragraph{Initialization Section}
\begin{listing}[conf]
[RunControl]
# The Ex0RunControl does not need any paramters.
\end{listing}

\paragraph{Configuration Section}
\begin{listing}[conf]
[RunControl]
EX0_STOP_RUN_AFTER_N_SECONDS = 60
\end{listing}

\subsubsection{LogCollector}
\label{sec:logcollector}
It is recommended to start the Log Collector directly after having started the Run Control and before starting other processors in order to collect all log messages generated by all other processes.

There are also two versions of the Log Collector.
The graphical version is called \texttt{euLog},
and the text-based version is called \texttt{euCliLogger}.

The command line pattern to startup a Log Collector is:
\begin{listing}[mybash]
$[euLog]$ -r tcp://{run_contorl_hostname}:{run_contorl_port} -a tcp://{listening_port}
\end{listing}

\begin{description}
\ttitem{-r \param{runcontrol\_addr}}
optional, \texttt{run\_control\_hostname} default value: localhost;  \texttt{run\_contorl\_port}  default value: 44000.
\ttitem{-a \param{listening\_addr}}
optional, \texttt{listening\_port} default value is 44002.
\end{description}

For this example setup, we will startup the GUI version of Log Collector, connect it to the Run Contorl at local port 44000 and make it serve at TCP port 44002:\\
\begin{listing}[mybash]
$[euLog]$ -r tcp://localhost:44000 -a tcp://44002
\end{listing}

After executing the above command, a new GUI window, as shown in \autoref{fig:LogCollector}, is opened and the RunControl displays a new connection. 
\begin{figure}[htb]
  \begin{center}
    \includegraphics[width=\textwidth]{src/images/eulog_ui}
    \caption{The Log Collector graphical user interface.}
    \label{fig:LogCollector}
  \end{center}
\end{figure}


\paragraph{Initialization Section}
\begin{listing}[conf]
[LogCollector.log]
# Currently, all LogCollectors have a hardcoded runtime name: log
EULOG_GUI_LOG_FILE_PATTERN = myexample_$12D.log
# the $12D will be converted a data/time string with 12 digits. 
# the file path is allowed add as prefix in this file name pattern,
# otherwise the log file is saved in working folder.
\end{listing}

\paragraph{Configuration Section}
\begin{listing}[conf]
[LogCollector.log]
# Currently, all LogCollectors have a hardcoded runtime name: log
# nothing
\end{listing}

\subsubsection{DataCollector}
\label{sec:datacollector}
There is only a text-based version called \texttt{euCliCollector}.
The command line pattern to startup a DataCollector is:
\begin{listing}[mybash]
$[euCliCollector]$ -n {code_name} -t {runtime_name} -r tcp://{run_control_hostname}:{run_contorl_port} -a tcp://{listening_port}
\end{listing}

\begin{description}
\ttitem{-n \param{code\_name}}
required.
\ttitem{-t \param{runtime\_name}}
required.
\ttitem{-r \param{runcontrol\_addr}}
optional, \texttt{run\_control\_hostname} default value: localhost;  \texttt{run\_contorl\_port}  default value: 44000.
\ttitem{-a \param{listening\_addr}}
optional, \texttt{listening\_port} default value is random.
\end{description}

By default, an example DataCollector \texttt{Ex0TgDataCollector} is available with the standard installation of EUDAQ.
For this example setup, we will startup two instances of \texttt{Ex0TsDataCollector} with runtime names \texttt{my\_dc} and \texttt{another\_dc}\\
\begin{listing}[mybash]
$[euCliCollector]$ -n Ex0TgDataCollector -t my_dc -r tcp://localhost:44000 -a tcp://45001
\end{listing}

\paragraph{Initialization Section}
\begin{listing}[conf]
[DataCollector.my_dc]
# nothing
\end{listing}

\paragraph{Configuration Section}
\begin{listing}[conf]
[DataCollector.my_dc]
EUDAQ_MN=my_mon
# send assambled event to the monitor with runtime name my_mon;
EUDAQ_FW=native
# the format of data file
EUDAQ_FW_PATTERN=$12D_run$6R$X
# the name pattern of data file
# the $12D will be converted a data/time string with 12 digits.
# the $6R will be converted a run number string with 6 digits.
# the $X will be converted the suffix name of data file.
# the file path is allowed add as a prefix to this name pattern,
# otherwise the data file is saved in working folder.
\end{listing}

\subsubsection{Producer}
\label{sec:testproducer}
There is only a text-based version called \texttt{euCliProducer}.
The command line pattern to startup a Producer is:
\begin{listing}[mybash]
$[euCliProducer]$ -n {code_name} -t {runtime_name} -r tcp://{run_control_hostname}:{run_contorl_port}
\end{listing}

\begin{description}
\ttitem{-n \param{code\_name}}
required.
\ttitem{-t \param{runtime\_name}}
required.
\ttitem{-r \param{runcontrol\_addr}}
optional, \texttt{run\_control\_hostname} default value: localhost;  \texttt{run\_contorl\_port}  default value: 44000.
\end{description}

By default, an example Producer \texttt{Ex0Producer} is available with the standard installation of EUDAQ.
For this example setup, we will startup two instances of \texttt{Ex0Producer} with runtime names \texttt{my\_pd0} and \texttt{my\_pd0}\\
\begin{listing}[mybash]
$[euCliProducer]$ -n Ex0Producer -t my_pd0 -r tcp://localhost:44000
$[euCliProducer]$ -n Ex0Producer -t my_pd1 -r tcp://localhost:44000
\end{listing}

\paragraph{Initialization Section}
\begin{listing}[conf]
[Producer.my_pd0]
EX0_DEV_LOCK_PATH = /tmp/mydev0.lock

[Producer.my_pd1]
EX0_DEV_LOCK_PATH = /tmp/mydev1.lock
\end{listing}

\paragraph{Configuration Section}
\begin{listing}[conf]
[Producer.my_pd0]
EUDAQ_DC=my_dc
# send events to the producer with runtime name my_dc.
# it is allowed to have a configure line as "EUDAQ_DC=his_dc,her_dc"
# to make the producer send events to muiltiple DataCollectors.  
EX0_PLANE_ID=0
EX0_DURATION_BUSY_MS=1
EX0_ENABLE_TRIGERNUMBER=1

[Producer.my_pd1]
EUDAQ_DC=my_dc
# send events to the producer with runtime name my_dc.
EX0_PLANE_ID=1
EX0_DURATION_BUSY_MS=1
EX0_ENABLE_TRIGERNUMBER=1
\end{listing}

\subsubsection{Monitor}
\label{sec:onlinemonitor}
There is a text-based version called \texttt{euCliMonitor}.
The command line pattern is:
\begin{listing}[mybash]
$[euCliMonitor]$ -n {code_name} -t {runtime_name} -r tcp://{run_control_hostname}:{run_contorl_port} -a tcp://{listening_port}
\end{listing}
\begin{description}
\ttitem{-n \param{code\_name}}
required.
\ttitem{-t \param{runtime\_name}}
required.
\ttitem{-r \param{runcontrol\_addr}}
optional, \texttt{run\_control\_hostname} default value: localhost;  \texttt{run\_contorl\_port}  default value: 44000.
\ttitem{-a \param{listening\_addr}}
optional, \texttt{listening\_port} default value is random.
\end{description}

By default, an example Monitor \texttt{Ex0Monitor} is available with the standard installation of EUDAQ. To simplify this example, the \texttt{Ex0Monitor} has no graphic windows, it can only print Event in commnad line terminal.
For this example setup, we will startup two instances of \texttt{Ex0Producer} with runtime names \texttt{my\_mon}\\
\begin{listing}[mybash]
$[euCliMonitor]$ -n Ex0Monitor -t my_mon -r tcp://localhost:44000 -a tcp://45002
\end{listing}

\paragraph{Initialization Section}
\begin{listing}[conf]
[Monitor.my_mon]
# nothing
\end{listing}

\paragraph{Configuration Section}
\begin{listing}[conf]
[Monitor.my_mon]
EX0_ENABLE_PRINT=0
EX0_ENABLE_STD_PRINT=0
EX0_ENABLE_STD_CONVERTER=1
\end{listing}

\subsection{Operating}
Once all the processes have been started, the RunControl GUI window will be as in \autoref{fig:RunControl_operate}.
\begin{figure}[htb]
  \begin{center}
    \includegraphics[width=0.8\textwidth]{src/images/eurun_ui_connected}
    \caption{The RunControl with all connections of the example setup.}
    \label{fig:RunControl_operate}
  \end{center}
\end{figure}

All of the processes report their status to the RunControl.
Depending on their status, EUDAQ and all processes can be initialized, configured or re-configured, data taking (runs) can be started and stopped, and the software can be terminated.

\begin{itemize}
\item First, the appropriate initialization file should be selected (see \autoref{sec:ConfigFiles} for creating and editing init-files). Then the \texttt{Init} button can be pressed,
which will send an initialization command to all connected processes.

\item Second, the appropriate configuration should be selected 
  (see \autoref{sec:ConfigFiles} for creating and editing configurations).
Then the \texttt{Config} button can be pressed,
which will send a configuration command
(with the contents of the selected configuration file) to all connected processes.
so that this information is always available along with the data.
\item Once all connected processes are fully configured, a run may be started, by pressing the \texttt{Start} button.
Whatever text is in the corresponding text box (``\texttt{Run:}'') when the button is pressed
will be stored as a comment in the data file.
This can be used to help identify the different runs later.
\item Once a run is completed, it may be stopped by pressing the \texttt{Stop} button.
\item At any time, a message may be sent to the log file by filling in the (``\texttt{Log:}'') text box and pressing the corresponding button.
The text should appear in the LogCollector window, and will be stored in the log file for later access.
\item Once the run is stopped, the system may be reconfigured with a different configuration, or another run may be started; or EUDAQ can be terminated.
  
\end{itemize}


\subsection{After data taking}
By default, EUDAQ provides tools to manage the the data saved in file by native EUDAQ format. Those tools could also be examples for users to write new code for more specific purpose. 
\subsubsection{Dump data}
\label{sec:dumpafterdatatacking}
To dump print Event from data file, the tool \texttt{euCliReader} is provided. The command line pattern is:
The command line pattern is:
\begin{listing}[mybash]
$[euCliReader]$ -i {input_file} -e {event_number_begin} -E {event_number_end} -tg {trigger_number_begin} -TG {tigger_number_end} -ts {timestamp_begin} -TS {timestamp_end} -s -std
\end{listing}
\begin{description}
\ttitem{-i \param{input\_file}}
required, the path of the input data file
\ttitem{-e \param{event\_number\_begin}}
optional, the low limit of event number to be printed 
\ttitem{-E \param{event\_number\_end}}
optional, the high limit of event number to be printed 
\ttitem{-tg \param{trigger\_number\_begin}}
optional, the low limit of trigger number to be printed 
\ttitem{-TG \param{trigger\_number\_end}}
optional, the high limit of trigger number to be printed 
\ttitem{-ts \param{timestamp\_begin}}
optional, the low limit of timestamp to be printed 
\ttitem{-TS \param{timestamp\_end}}
optional, the high limit of timestamp to be printed
\ttitem{-s}
optional, enable the print of statistics 
\ttitem{-std}
optional, enable the Standard Event Converter and print out StdEvent
\end{description}

The option pairs \texttt{-e -E}, \texttt{-tg -TG} and \texttt{-ts -TS} apply range limites and pick up the most intreasting Event from data file. If an option pair is not specified by user, there will be not range limit for this option pair.

\subsubsection{Convert data format}
\label{sec:convertafterdatatacking}
To convert Event from data file, the tool \texttt{euCliConverter} is provided. The command line pattern is:
The command line pattern is:
\begin{listing}[mybash]
$[euCliConverter]$ -i {input_file} -o {output_file} -ip
\end{listing}
\begin{description}
\ttitem{-i \param{input\_file}}
required, the path of the input data file
\ttitem{-o \param{output\_file}}
required, the path of the output data file. 
\ttitem{-ip}
optional, enable the print of input Event 
\end{description}

If the output file has the suffix \texttt{slcio} and LCIO feature of EUDAQ is enabled at compiling time, it will generate LCIO data file.

% !TEX root = EUDAQUserManual.tex
\section{Integration with User Hardware}\label{sec:Integration}
EUDAQ itself is only a data taking framework. It means that the users with their dedicated hardware and readout software are required to write some code to bridge the hardware specific readout software to the EUDAQ framework. The minimum adaptation task is to write a Producer for each piece of hardware, a Data Collector to receive the data (a.k.a.\ Event) from the Producers. 

\subsection{Announcement of Derived Class}
The derived EUDAQ classes provided by the user will be compiled and packed to a dynamic shared library (EUDAQ Module Library). At compiling/linking time of EUDAQ core library, it does not know of the existence of any Module Library. When the EUDAQ core library is being loaded by any application, the core library will look for any library file with the name prefix libeudaq\_module\_ in the module folder. All pattern matched libraries will be loaded. It is the point at which each derived EUDAQ class announces itself to the EUDAQ runtime environment.

Technically, the announcement of a derived class is done by a call to the correlated static function provided by a generic C++ template (eudaq::Factory). \autoref{tab:derivable} is the list of derivable EUDAQ classes.

\begin{table}
\centering
\small
\begin{tabular}{ l | l }
  \textbf{Class} & \textbf{Description}\\
  \hline
  \texttt{eudaq::Producer} & Sec. \ref{sec:ProducerWriting}\\
  \texttt{eudaq::DataCollector} & Sec. \ref{sec:DataCollectorWriting}\\
  \texttt{eudaq::RunControl} & Sec. \ref{sec:RunControlWriting}\\
  \texttt{eudaq::Event} &  Sec. \ref{sec:Event} \\
  \texttt{eudaq::LogCollector} & Sec. \ref{sec:logcollector} \\
  \texttt{eudaq::Monitor} & Sec. \ref{sec:MonitorWriting}\\
  \texttt{eudaq::FileWriter} & Sec. \ref{sec:FileWriterWriting}\\
  \texttt{eudaq::FileReader} & Sec. \ref{sec:FileReaderWriting}\\
  \texttt{eudaq::StdEventConverter} & Sec. \ref{sec:DataConverter} \\
  \texttt{eudaq::LCEventConverter} & Sec. \ref{sec:DataConverter} \\
  \texttt{eudaq::TransportServer} & internal only \\
  \texttt{eudaq::TransportClient} & internal only \\
\end{tabular}
\caption{Derivable Classes.}
\label{tab:derivable}
\end{table}

\subsection{Serializable}
As a distributed DAQ framework, a runtime setup of the EUDAQ system include several applications. Data objects will go through the boundary of an application or a computer. Those data objects should have the capability to be serialized. When a data object is serialized, all the crucial data of this data object is fed to serialized memory which then can be sent by plain binary to another application and reconstructed as a copy of the original data object. \\

All the classes which hold serializable data are derived from a base serializable class (eudaq::Serializable). All serializable data class
should implement the function \texttt{Serialize} which serializes the inner data object and feeds an eudaq::Serializer, and a constructor function which takes the reference of eudaq::Deserializer as input parameter.\\

\autoref{tab:serializable} is the list of serializable EUDAQ classes.

\begin{table}
\centering
\small
\begin{tabular}{ l | l }
  \textbf{Class} & \textbf{Description}\\
  \hline
  \texttt{eudaq::Event} & Sec. \ref{sec:Event} \\
  \texttt{eudaq::Configuration} & Contains configuration information \\
  \texttt{eudaq::LogMessage} & Messages reported to the LogCollector \\
  \texttt{eudaq::Status} & States reported to the RunControl \\
\end{tabular}
\caption{Serializable Classes.}
\label{tab:serializable}
\end{table}

\subsubsection{Event}\label{sec:Event}
eudaq::Event is most important serializable class which holds physics data from the hardware. The Producer is the EUDAQ component which creates the object eudaq::Event and feeds it the physics data from measurements.  \autoref{tab:eventdata} lists the variables inside the eudaq::Event. \\

\begin{table}
\centering
\small
\begin{tabular}{ l | l | l }
  \textbf{variable} & \textbf{C++ type} & \textbf{Description}\\
  \hline
  \texttt{m\_type} & \texttt{uint32\_t} & event type\\
  \texttt{m\_version} & \texttt{uint32\_t} & version\\
  \texttt{m\_flags} & \texttt{uint32\_t} & flags\\
  \texttt{m\_stm\_n} & \texttt{uint32\_t} & device/stream number\\
  \texttt{m\_run\_n} & \texttt{uint32\_t} & run number\\
  \texttt{m\_ev\_n} & \texttt{uint32\_t} & event number\\
  \texttt{m\_tg\_n} & \texttt{uint32\_t} & trigger number\\
  \texttt{m\_extend} & \texttt{uint32\_t} & reserved word\\
  \texttt{m\_ts\_begin} & \texttt{uint64\_t} & timestamp at the begin of event\\
  \texttt{m\_ts\_end} & \texttt{uint64\_t} & timestamp at the end of event\\
  \texttt{m\_dspt} & \texttt{std::string} & description\\
  \texttt{m\_tags} & \texttt{std::map<std::string, std::string>} & tags\\
  \texttt{m\_blocks} & \texttt{std::map<uint32\_t, std::vector<uint8\_t>>} & blocks of raw data\\
  \texttt{m\_sub\_events} & \texttt{std::vector<EventSPC>} & pointers of sub events\\
\end{tabular}
\caption{Variables of eudaq::Event.}
\label{tab:eventdata}
\end{table}

The \texttt{m\_blocks} is physics data which can only be known by the user who owns the hardware. There is a pair of timestamps to define the time slice when the physics event occurs, and a trigger number to identify the trigger sequence. Timestamps and trigger number are optional to be set if you are going to use them to synchronize data from multiple stream/device. It is also possible to have sub events inside an eudaq::Event object. The sub eudaq::Event objects are held by std::shared\_ptr, see the next section.

\subsection{Ownership}
std::shared\_ptr and std::unique\_ptr are heavily used in EUDAQ to hold the object pointers of the serializable class and the derivable class.  They get rid of the unnecessary and ineffective memory copy and are exception safe for any memory leakage.

\subsection{Command/Status Handling}
\subsubsection{RunControl and CommandReceiver}\label{sec:runcontrol}
\paragraph{RunControl} eudaq::RunControl is the command sender which issues commands according to user actions from the GUI or CLI.
\paragraph{CommandReceiver} eudaq::Producer, eudaq::DataCollector, eudaq::LogCollector and eudaq::Monitor are command receivers (eudaq::CommandReceiver) which executes the correlated function according to the command. The command receiver will set up a status (eudaq::Status) and report the status to RunControl.

\subsubsection{State Model}\label{sec:fsm}
\begin{figure}
\begin{center}
\includegraphics[width=0.9\textwidth]{src/images/fsmv2.pdf}
\end{center}
\caption{The FSM of EUDAQ.}
\label{fig:fsm}
\end{figure}

The \gls{FSM} is implemented in both the RunControl and the CommandReceiver (see \autoref{fig:fsm}) \cite{Shirokova:2016}. \\

Each CommandReceiver can always be characterized by the current state (\autoref{tab:statetab}).
\begin{table}
\centering
\small
\begin{tabular}{ l | l | l }
  \textbf{State} & \textbf{Enumerate Value} & \textbf{Acceptable Command} \\
  \hline
  \texttt{Error} & \texttt{eudaq::Status::STATE\_ERROR} & DoReset \\
  \texttt{Uninitialised} & \texttt{eudaq::Status::STATE\_UNINIT} & DoInitialise DoReset \\
  \texttt{Unconfigured} & \texttt{eudaq::Status::STATE\_UNCONF} & DoConfigure DoReset \\
  \texttt{Configured} & \texttt{eudaq::Status::STATE\_CONF} & DoConfigure DoStartRun DoReset \\
  \texttt{Running} & \texttt{eudaq::Status::STATE\_RUNNING} & DoStopRun \\
\end{tabular}
\caption{States of RunControl client.}
\label{tab:statetab}
\end{table}


The state of RunControl is determined by the lowest state of the connected client in the following priority: ERROR, UNINITIALISED, UNCONFIGURED, CONFIGURED, RUNNING. It means, for example, that even if only one connection is in the ERROR state, the whole machine will also be in that state. This prevents such mistakes as running the system before every component has finished the configuration.

\subsubsection{Command}\label{sec:command}

\begin{table}
\centering
\small
\begin{tabular}{ l | l | l | l |l }
  \textbf{Command} & \textbf{RunControl Side} & \textbf{Client Side} & \textbf{Pre State} & \textbf{State of Success}\\
  \hline
  \texttt{Initialise} & Initialise & DoInitialise & Uninitialised & Unconfigured\\
  \texttt{Configure} & Configure & DoConfigure & Unconfigured Configured & Configured\\
  \texttt{StartRun} & StartRun & DoStartRun & Configured & Running\\
  \texttt{StopRun} & StopRun & DoStopRun & Running & Configured\\
  \texttt{Reset} & Reset & DoReset & Error & Uninitialised\\
  \texttt{Terminate} & Terminate & DoTerminate & Any apart from Running & - \\
\end{tabular}
\caption{List of Commands.}
\label{tab:cmdtab}
\end{table}


\paragraph{Initialise}
When an EUDAQ process is in the state of Uninitialised, it accepts the Initialise command and executes the DoInitialise function provided by the user. When it is initialized, it does not accept further Uninitialised commands until it is reset to the Uninitialised state.

\paragraph{Configure}
When an EUDAQ process is in the state of Unconfigured or Configured, it accepts the Configure command and executes the DoConfigure function provided by the user. Please be aware that the second Configure command after a successful execution of the Configure command is allowed. It means the EUDAQ process is re-configurable but not re-initialisible.

\paragraph{StartRun}
When an EUDAQ process is in the state of Configured, it also accepts the StartRun command and executes the DoStartRun function provided by the user. In the case of the Producer, the Producer should talk to the hardware and produce and send Event data. In the case of the DataCollector, the DataCollector will get the Event stream from its connected Producer.

\paragraph{StopRun}
When an EUDAQ process is in the state of Running, it only accepts the StopRun command and executes the DoStopRun function provided by the user. RunControl will not send the StopRun command to the DataCollector until all Producers have returned from the DoStopRun function and feedback their status.

\paragraph{Reset}
In any case when an EUDAQ process goes to the state of Error, only the Reset command is allowed.

\paragraph{Terminate}
Except for the state of Running, the terminate command is allowed on all other states. When the Terminate command is called, the EUDAQ processes will be closed. If the user has to do something before exiting the EUDAQ process, this should be entered into the DoTerminate function which will then carry out the process before exiting.


\include{src/Producers}
% !TEX root = EUDAQUserManual.tex
\section{Writing a DataCollector}\label{sec:DataCollectorWriting}
A DataCollector can take data from Producers and merge/synchronise those data.
A DataCollector consists of a CommandReceiver and a DataReceiver, where the first receives commands from the Run Control while the latter receives binary data events from one or more Producers. A dedicated synchronisation method should be implemented inside of the DataCollector to pack the incoming data.
The DataCollector base class is provided in order to simplify the integration. Example code for Producers is provided.

\subsection{DataCollector Prototype}\label{sec:datacollector_hh}

\autoref{ls:datacoldef}, below, is part of the header file which declares the eudaq::Producer. You are required to write the user DataCollector derived from eudaq::DataCollector.
There are nine virtual methods, belonging to two categories, which should be implemented by the user. The first category includes the methods \lstinline[style=cpp]{DoInitialise},
\lstinline[style=cpp]{DoConfigure}, \lstinline[style=cpp]{DoStopRun}, \lstinline[style=cpp]{DoStartRun}, \lstinline[style=cpp]{DoReset} and \lstinline[style=cpp]{DoTerminate} which are called by command received and should return as soon as possible. The other category includes the methods \lstinline[style=cpp]{DoConnect}, \lstinline[style=cpp]{DoDisconnect}, and \lstinline[style=cpp]{DoReveive} which respond to a connection in establishing or deleting, or a new coming Event.

\lstinputlisting[label=ls:datacoldef, style=cpp, linerange=BEG*DEC-END*DEC]{../../main/lib/core/include/eudaq/DataCollector.hh}

The virtual function Exec() is optional to be implemented in user DataCollector. Internally, the base Exec() to create two threads for command execution and data receiving,  itself then goes to an infinite loop and can never return until the Terminate command is executed. In case you are going to implement it yourselves, please read the source code to find detailed information.

\subsection{Example Code: DirectSave}\label{sec:directsavedatacollector_cc}
Here is an example code of a derived DataCollector, named DirectSaveDataCollector. As its name hints,  it does nothing except save all incoming Event data to disk directly. Printing out of the Event to screen is optional for debugging.
\lstinputlisting[label=ls:datacoldirsave, style=cpp]{../../user/eudet/module/src/DirectSaveDataCollector.cc}

Two virtual methods are implemented in DirectSaveDataCollector, \lstinline[style=cpp]{DoConfigure()} and \lstinline[style=cpp]{DoReceive(eudaq::ConnectionSPC id, eudaq::EventUP ev)}. It registers itself to the correlated \lstinline[style=cpp]{eudaq::factory} by the hash number from the name string \lstinline[style=cpp]{DirectSaveDataCollector}. 

\subsection{Example Code: SyncTrigger}\label{sec:ex2datacollector_cc}
Now, a more realistic example. The full source code is available here, \autoref{ls:ex2datacoldec}. It can merge the eudaq::Event by trigger number from the connected eudaq::Producer.
\lstinputlisting[label=ls:ex2datacoldec, style=cpp]{../../user/example/module/src/Ex0TgDataCollector.cc}
Compared to previous DirectSaveDataCollector example, two more virtual methods are implemented. They are \lstinline[style=cpp]{DoConnect}, \lstinline[style=cpp]{DoDisconnect}. The first, \lstinline[style=cpp]{DoConnect}, will be called when a new connection from eudaq::Producer is created, and the other, \lstinline[style=cpp]{DoDisconnect}, will be called  when the connection is expired. The information of the correlated connection is provided by the incoming parameter. The lifetime of the connection between the eudaq::DataCollector and eudaq::Producer is a data-taking run. No \lstinline[style=cpp]{DoConfigure} method is implemented, and this DataCollector is not Configurable.










% !TEX root = EUDAQUserManual.tex
\section{Writing a Data Converter}\label{sec:DataConverter}
As a framework, EUDAQ itself is developed with no knowledge of how hardware data can be decoded. Only the user can know the specific details of the readout data from hardware, so users are required to write a data converter derived from eudaq::DataConverter to convert to another format. The converted data can then be stored on disk or used as input data for any non-EUDAQ software.\\

Currently, as historical legacy, two different formats are provided along with the native raw eudaq::Event. They are eudaq::StandardEvent for pixel detectors, and eudaq::LCEvent as an EUDAQ wrapper for the LCIO format.

\subsection{Event Structure}
eudaq::Event is the most important data container in the EUDAQ system. eudaq::Event should be filled by detector data properly in order to transmit among eudaq clients, e.g.\ Producer and DataCollector. \autoref{tab:eventdata} lists all the member variables inside of eudaq::event. Not all of these variables have to be filled. For example, in case there is a trigger number but no timestamp information, the m\_ts\_begin and m\_ts\_end can be left untouched.

%%\begin{table}[!h]
%%{\footnotesize
%%\begin{tabular}{l|l|p{2cm}|p{5.5cm}}
%%Variable & Type & Default Value & Comment \\
%% \hline
%% m\_type & uint32\_t & - & Event type\\
%% m\_version & uint32\_t & - & Version of EUDAQ\\
%% m\_flag & uint32\_t & - & Event flag\\
%% m\_stm\_n & uint32\_t & - & Stream/device number\\
%% m\_run\_n & uint32\_t & - & Run number\\
%% m\_ev\_n & uint32\_t & - & Event number\\
%% m\_tg\_n & uint32\_t & - & Trigger number\\
%% m\_extend & uint32\_t & - & Reserved word\\
%% m\_ts\_begin & uint64\_t & - & Timestamp at the begin of trigger\\
%% m\_ts\_end & uint64\_t & - & Timestamp at the end of trigger\\
%% m\_dspt & std::string & - & Description String\\
%% m\_tags & std::map$<$std::string, std::string$>$& - & Tag pairs\\
%% m\_blocks & std::map$<$uint32\_t, std::vector$<$uint8\_t$>>$ & - & Raw data blocks\\
%% m\_sub\_events & std::vector$<$EventSPC$>$& - & Sub events\\
%% \end{tabular}
%% \caption{All member variables in eudaq::Event.}
%% \label{tab:eventvarialbe}
%% }
%% \end{table}
%% 
The eudaq::Event maye also have sub events. 

\subsubsection{RawDataEvent}\label{sec:convrawdata}
eudaq::RawDataEvent is derived from eudaq::Event with the m\_type always set to eudaq::cstr2hash(``RawDataEvent''). It has the same member variables and functions as the base eudaq::Event. However the member variable m\_extend is used as an identification number of the sub type of event.

\subsubsection{StandardEvent}\label{sec:convstddata}
eudaq::StandardEvent is derived from eudaq::Event with the m\_type always set to eudaq::cstr2hash(``StandardEvent''). Historically, it presents a beam telescope tracking-hit event with all hit information of sensor planes. The hit information is contained by eudaq::StandardPlane. The beam telescope planes are pixel sensors. The spatial position and signal amplitude of the fired pixels are zero-compressed inside eudaq::StandaredPlane.

\subsubsection{TTreeEvent}
\label{sec:RawEvent2TTreeEventConverter}

Another working example of converting the \texttt{raw} event to \texttt{ROOT TTree} format is also provided. The details and flow of converter is described in Annexe \ref{sec:TTreeConverter}.



\subsection{Example Code: RawEvent2StdEvent}\label{sec:Ex0RawEvent2StdEventConverter_cc}
This example DataConverter is named Ex0RawEvent2StdEventConverter. As indicated by the name, it converts the eudaq::RawDataEvent to eudaq::StandardEvent. The sub type of eudaq::RawDataEvent is ``my\_ex0'' which is also used to calculate the hash and register it to the eudaq::Factory. If an eudaq::RawDataEvent object announcing its sub-type by ``my\_ex0'' exists when doing the data converting, this object will be forwarded to that Ex0RawEvent2StdEventConverter.
\lstinputlisting[label=ls:ex0raw2std, style=cpp]{../../user/example/module/src/Ex0RawEvent2StdEventConverter.cc}




 

\include{src/Monitor}
\include{src/RunControl}
\include{src/NewData}
%% \include{src/OtherParts}
% !TEX root = EUDAQUserManual.tex
\section{Contributing to EUDAQ}
\label{sec:contributing}

\subsection{Reporting Issues}
\label{sec:reporting}
The GitHub server, on which EUDAQ is hosted, provides a system for reporting bugs and for requesting new features.
It is accessible at the following address: 

\url{https://github.com/eudaq/eudaq/issues}.

Here you may submit new reports (you are required to register first to do this),
or follow the status of existing bugs and feature requests.
This is recommended over (or at least, as well as) sending an email to the developers,
as it ensures a record of the issue is available, and others may follow the progress.

\subsection{Regression Testing}
\label{sec:ctest}

to be re-activated for EUDAQ2.

%If a CMake version later than 3.1 is found and Python is installed
%together with the \texttt{numpy} package, several regression tests are made
%available that can be executed through CTest. The tests are based on
%the Python wrapper around EUDAQ components as described in
%section~\ref{sssec:pywrapper}. Run the tests by typing
%
%\begin{listing}[mybash]
%  cd build
%  cmake ..
%  ctest
%\end{listing}
%
%This starts the script \texttt{etc/tests/run\_dummydataproduction.py}
%which runs a short DAQ session with instances of RunControl,
%DataCollector and a (dummy) Producer and compares the output to a
%reference file stored on AFS at DESY. If your system is set up
%correctly, you have access to the reference file, and the basic
%components of the EUDAQ library work, all tests should pass.
%To see the output of failing tests, you can add the
%\texttt{--output-on-failure} parameter to the CTest command.
%
%These basic tests can easily be extended to test other parts of the
%core framework or of your own producer. Take a look at the
%\texttt{etc/tests/testing.cmake} CMake script and the central
%\texttt{CMakeLists.txt} file where it is included for an example of
%how to set up tests with CTest.
%
%The automated nightly tests are defined in CMake scripts located in
%\texttt{etc/tests/nightly} and are executed by the scripts
%\texttt{run\_nightly.sh} and \texttt{run\_nightly.bat} for Unix and
%Windows platforms, respectively. In addition to the dummy run
%described above, the nightly tests check out all changes from the
%central repository, build the full code base, and submit all results
%to the CDash webserver hosted at DESY: \url{http://aidasoft.desy.de/CDash/index.php?project=EUDAQ}

\subsection{Commiting Code to the Main Repository}
\label{sec:commiting}



If you would like to contribute your code back into the main repository, please follow the ``fork \& pull request'' strategy:

\begin{itemize}
\item Create a user account on github, log in
\item ``Fork'' the (main) project on github (using the button on the page of the main repo)
\item \emph{Either}: clone from the newly forked project and add
  'upstream' repository to local clone (change user names in URLs
  accordingly):
  \begin{listing}[mybash]
git clone https://github.com/your_account_name/eudaq eudaq
cd eudaq
git remote add upstream https://github.com/eudaq/eudaq.git
\end{listing}
\item \emph{or} if edits were made to a previous checkout of upstream: rename origin to upstream, add fork as new origin:

  \begin{listing}[mybash]
cd eudaq
git remote rename origin upstream
git remote add origin https://github.com/your_account_name/eudaq
git remote -v show
\end{listing}
\item Optional: edit away on your local clone! But keep in sync with
  the development in the upstream repository by running
  \begin{listing}
git fetch upstream        # download named heads or tags
git pull upstream master  # merge changes into your branch
\end{listing}
on a regular basis. Replace \texttt{master} by the appropriate branch if you work on a separate one.
Don't forget that you can refer to issues in the main repository anytime by using the string \texttt{eudaq/eudaq\#XX} in your commit messages, where \texttt{XX} stands for the issue number, e.g.
  \begin{listing}[mybash]
[...]. this addresses issue eudaq/eudaq#1
\end{listing}
\item Push the edits to origin (our fork)
  \begin{listing}[mybash]
git push origin
\end{listing}
(defaults to \texttt{git push origin master} where origin is the repo and master the branch)
\item Verify that your changes made it to your github fork and then click there on the ``compare \& pull request'' button
\item Summarize your changes and click on ``send''
\item Thank you!
\end{itemize}


Working together on a branch:
If you have a copy installed, and want to update it to the
latest version, you do not need to clone the repository again, just change to the \texttt{eudaq} directory use the command:
\begin{listing}[mybash]
git pull
\end{listing}
to update your local copy with all changes commited to the central repository.%



% Appendices
\appendix
\include{src/PlatformIssue}
%%%% \include{src/CompilingOnWindows}

\printglossaries

\section*{Acknowledgements}
This project receives funding from the European Union’s Horizon 2020 Research and Innovation programme under Grant Agreement no. 654168.
The support is gratefully acknowledged.
\emph{Disclaimer}: The information herein only reflects the views of its authors and not those of the European Commission and no warranty expressed or implied is made with regard to such information or its use.

Before, this work was supported by the Commission of the European Communities under the 6\textsuperscript{th} Framework Programme ``Structuring the European Research Area,'' contract number RII3--026126, and received funding from the European Commission under the FP7 Research Infrastructures project AIDA, grant agreement no. 262025. 


%\bibliographystyle{unsrt}
\bibliographystyle{src/ClassicCite}
\bibliography{src/References}

\end{document}
